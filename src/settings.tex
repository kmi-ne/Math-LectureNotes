\makeatletter

%%%%% \section{真理値と条件分岐}
\usepackage{ifthen}
\NewDocumentCommand{\SetBool}{m !t+ !t-}{
  \IfBooleanT{#2}{\setboolean{#1}{true}}{\setboolean{#1}{false}}
}
\NewDocumentCommand{\SetNewBool}{m !t+ !t-}{
  \newboolean{#1}
  \IfBooleanT{#2}{\setboolean{#1}{true}}{\setboolean{#1}{false}}
}
\newcommand{\IfBoolTF}[3]{\ifthenelse{\boolean{#1}}{#2}{#3}}
\newcommand{\IfBoolT}[2]{\IfBoolTF{#1}{#2}{}}
\newcommand{\IfBoolF}[2]{\IfBoolTF{#1}{}{#2}}
% - - - ページの偶奇による条件分岐 - - -
\usepackage{ifoddpage}
\newcommand{\IfPageLR}[2]{\checkoddpage\ifoddpage #2\else #1\fi}
% - - - 値設定 - - -
\MyBoolSettings



%%%%% \section{ページレイアウト}
\setpagelayout+{
  headheight=20\jsc@mpt,
  hscale=0.8,
  hmarginratio=1:1,
  vscale=0.9,
  vmarginratio=1:1
}
\geometry{marginparwidth=20pt}

\usepackage{multicol}



%%%%% \section{フォント/書体/文字関連}
% - - - 和文 - - -
\ltjsetparameter{jacharrange={-2,-3}}
\setsansjfont{HaranoAjiGothic}[
  UprightFont={*-Medium},
  BoldFont={*-Bold},
]
% - - - 欧文/数学記号 - - -
\PassOptionsToPackage{\MyMathStyle}{unicode-math}
\usepackage[olddefault, newcmbb, varnothing]{fontsetup}
\setmathfont[range={scr, bfscr}, StylisticSet=1]{NewCMMath-Regular}
\DeclareMathAlphabet{\cl}{OMS}{cmsy}{m}{n} % 旧チャンセリー筆記体
\setsansfont{nimbussans}
% - - - テキスト書体 - - -
\renewcommand{\sf}{\sffamily}
\newcommand{\sfbf}{\sffamily\bfseries}
\newcommand{\textsfbf}[1]{{\sffamily\bfseries #1}}
% - - - 追加記号 - - -
\usepackage{manfnt}
\usepackage[old]{old-arrows}
\usepackage{fontawesome5}
% - - - Unicode - - -
\usepackage{newunicodechar}
% - - - ルビ - - -
\usepackage{luatexja-ruby}



%%%%% \section{色}
\usepackage{xcolor, luacolor, lua-ul}
% - - - 色を定義 - - -
% \definecolor{defaultcolor}{HTML}{333333}
% \color{defaultcolor}
% \colorlet{black}{defaultcolor}
\definecolor{mygreen}{HTML}{75b651}
\definecolor{myblue}{HTML}{6e9dc2}
\definecolor{myred}{HTML}{c26e7c}
\definecolor{myyellow}{HTML}{b8ba05}
\definecolor{mygray}{HTML}{878787}
\definecolor{myemphcolor}{HTML}{d94876}
\definecolor{myurlcolor}{HTML}{ad0d00}
\definecolor{mylinkcolor}{HTML}{2a4f99}
\definecolor{mycitecolor}{HTML}{2a4f99}
% - - - テキスト彩色用マクロ - - -
\newcommand{\赤}[1]{\textcolor{myred}{#1}}
\newcommand{\青}[1]{\textcolor{myblue}{#1}}
\newcommand{\緑}[1]{\textcolor{mygreen}{#1}}
\newcommand{\黄}[1]{\textcolor{myyellow}{#1}}
\newcommand{\灰}[1]{\textcolor{mygray}{#1}}



%%%%% \section{ヘッダー/フッター}
% クラスファイルによるページスタイル強制変更を除去
\renewcommand{\plainifnotempty}{}
% - - - 部・章・節番号の出力形式を設定 - - -
\renewcommand{\postchaptername}{講}
\renewcommand{\presectionname}{§}
\newcommand{\presubsectionname}{§}
\newcommand{\章番号}{\@chapapp\thechapter\@chappos}
\newcommand{\節番号}{\@secapp\thesection\@secpos}
\newcommand{\小節番号}{\presubsectionname\thesubsection}
% - - - 専用ページスタイルの定義 - - -
\usepackage{fancyhdr}
\fancypagestyle{plain}{%
  \fancyhf{}
  \fancyheadinit{\small\sf}
  \fancyhead[RO,LE]{\textit{\thepage}}
  \fancyhead[CE]{\LeftHeader}
  \fancyhead[CO]{\RightHeader}
}



%%%%% \section{余白}
\usepackage{marginnote}
% - - - 余白部分のアイコン定義 - - -
% 定義用マクロ
\newcommand{\NewRangeIcon}[2]{
  \newenvironment{#1}
  {\IfPageLR
    {\marginnote[\hfill#2]{}}
    {\reversemarginpar\marginnote[\hfill#2]{}\normalmarginpar}%
  }
  {\IfPageLR
    {\reversemarginpar\marginnote{#2\hfill}\normalmarginpar}
    {\marginnote{#2\hfill}}%
  }
}

% 定義
\NewRangeIcon{dng}{\dbend}



%%%%% \section{見出しとセクショニング}
% \part
\patchcmd{\part}
  {\thispagestyle{empty}}
  {\pagestyle{empty}}
  {}{}
% → tcolorbox
% - - - セパレーター - - -
\usepackage{froufrou}
\newcommand{\=====}{\froufrou*}



%%%%% \section{脚注}
\RenewDocumentCommand{\*}{sO{}}{\IfBlankTF{#2}{\footnotemark{}}{\footnotetext{#2}}}



%%%%% \section{目次}
\patchcmd{\tableofcontents}{*{\contentsname}}{*{\contentsname}\addcontentsline{toc}{chapter}{\contentsname}}{}{}
\setcounter{tocdepth}{2}
\patchcmd{\l@subsection}{3.5\jsZw}{4\jsZw}{}{}



%%%%% \section{ウォーターマーク/ラベル表示}
\IfBoolT{ドラフト出力}{
  \usepackage{draftwatermark}
  \usepackage[notref]{showkeys}
  \renewcommand*{\showkeyslabelformat}[1]{\normalfont\scriptsize\ttfamily\llap{\fbox{#1}}}
}



%%%%% \section{ハイパーリンク}
\usepackage{hyperref}
\hypersetup{
  setpagesize=false,
  bookmarksdepth=3,
  bookmarksnumbered=true,
  colorlinks=true,
  linkcolor=mylinkcolor,
  citecolor=mycitecolor,
  urlcolor=myurlcolor,
  pdfauthor={鴎海},
  linktoc=all
}
\NewCommandCopy{\oldhref}{\href}
\renewcommand{\href}[2]{\oldhref{#1}{\underLine[myurlcolor]{#2}}}
% - - - namerefで参照可能なラベル - - -
% \mylabel{#1:label}"#2:nameref"
\NewDocumentCommand{\mylabel}{mD""{}}{%
  \bgroup
    \protected@edef\@currentlabelname{#2}
    \label{#1}
  \egroup%
}
% - - - 任意の位置のラベル - - -
\newcommand{\labelhere}[1]{\phantomsection\label{#1}}
% - - - ページ番号が右上に表示されたハイパーリンク - - -
% \seepage[#1:label]{#2:テキスト}
\newcommand{\seepage}[2]{\hyperref[#1]{#2\textsuperscript{→p.~\pageref*{#1}}}}



%%%%% \section{cleveref}
\usepackage[nameinlink]{cleveref}
% - - - ページ - - -
\crefname{page}{p.}{pp.}
% - - - 部 - - -
\crefformat{part}{#2\prepartname #1\postpartname #3}
% - - - 章 - - -
\crefformat{chapter}{#2\@chapapp #1\@chappos #3}
% - - - 節 - - -
\crefname{section}{\presectionname}{\presectionname}
% - - - 小節 - - -
\crefname{subsection}{\presubsectionname}{\presubsectionname}
% - - - 数式 - - -
\crefname{equation}{式}{式}
\creflabelformat{equation}{#2(#1)#3}



%%%%% \section{リスト}
\usepackage{tasks}
\usepackage[inline]{enumitem}
\setlist{align=left, leftmargin=*}
% - - - thmlist 環境: theorem list - - -
% tcolorbox 内の項目のための enumerate 類似環境
\newlist{thmlist}{enumerate}{2}
\setlist[thmlist, 1]{label=\arabic{thmlisti}., ref=\thetcbcounter.\arabic{thmlisti}}
\setlist[thmlist, 2]{label=\arabic{thmlisti}.\arabic{thmlistii}., ref=\thetcbcounter.\arabic{thmlisti}.\arabic{thmlistii}}
% - - - myenum 環境: my enumerate - - -
\newlist{myenum}{enumerate}{3}
\setlist[myenum, 1]{label=(\arabic{myenumi}), ref=(\arabic{myenumi})}
\setlist[myenum, 2]{label=(\arabic{myenumi}.\arabic{myenumii}), ref=(\arabic{myenumi}.\arabic{myenumii})}
\setlist[myenum, 3]{label=(\arabic{myenumi}.\arabic{myenumii}.\arabic{myenumiii}), ref=(\arabic{myenumi}.\arabic{myenumii}.\arabic{myenumiii})}
% - - - \item を ref と nameref で参照可能にしたもの- - -
% \begin{...}
%   \item[#1:項目名](#2:label)"#3:nameref"
% \end{...}
\NewCommandCopy{\olditem}{\item}
\RenewDocumentCommand{\item}{O{}D(){}D""{}}{%
  \IfBlankTF{#1}
    {\olditem
    \IfBlankTF{#3}
      {\IfBlankF{#2}{\mylabel{#2}}}
      {\renewcommand{\@currentlabel}{#3}\mylabel{#2}"#3"}%
    }
    {\olditem[#1]
    \phantomsection
    \IfBlankTF{#3}
      {\renewcommand{\@currentlabel}{#1}\mylabel{#2}"#1"}
      {\renewcommand{\@currentlabel}{#3}\mylabel{#2}"#3"}%
    }
}



%%%%% \section{コード}
\usepackage{../formalproofwriter}



%%%%% \section{数式}
\everymath=\expandafter{\the\everymath\narrowbaselines\displaystyle}
\usepackage{leftidx}
\usepackage{longmath}
\usepackage{scalerel}
\allowdisplaybreaks[4]



%%%%% \section{tcolorbox}
\usepackage[skins, breakable]{tcolorbox}

% - - - 見出し - - -
% chapter
\newtcolorbox{chapbox}{
  phantom={\phantomsection},
  tile,
  fontupper=\sf,
  halign=flush center,
}
\RenewDocumentCommand{\chapter}{sm!t+}{
  \clearpage
  \def\章タイトル{#2}
  \IfBooleanF{#1}{\refstepcounter{chapter}}
  \begin{chapbox}
    \IfBooleanF{#1}{\Large ─\ \章番号\ ─\\}
    \LARGE \章タイトル
  \end{chapbox}
  \IfBooleanTF{#1}{%
    \IfBooleanT{#3}{\addcontentsline{toc}{chapter}{\章タイトル}}
    \def\LeftHeader{\章タイトル}
    \def\RightHeader{\章タイトル}
  }{%
    \addcontentsline{toc}{chapter}{\protect\numberline{\章番号}\章タイトル}
    \def\LeftHeader{\章番号 \quad \章タイトル}
    \def\RightHeader{\章番号 \quad \章タイトル}
  }%
  \protected@edef\@currentlabelname{\章タイトル}
  \pagestyle{plain}
  \thispagestyle{empty}
  \addvspace{2\baselineskip}
}
% section
\newtcolorbox{sectbox}{
  phantom={\phantomsection}, 
  blank,
  size=minimal,
  fontupper=\sf,
}
\RenewDocumentCommand{\section}{sm!t+}{
  \addvspace{1.5\baselineskip}
  \def\節タイトル{#2}
  \IfBooleanF{#1}{\refstepcounter{section}}
  \begin{sectbox}
    \IfBooleanTF{#1}
      {\Large #2\hspace{.5em}\hrulefill}
      {\large \節番号\hspace{.5em}\hrulefill \vspace{.5ex}\\ \Large #2}
  \end{sectbox}
  \IfBooleanTF{#1}{%
    \IfBooleanT{#3}{\addcontentsline{toc}{section}{\節タイトル}}
    \def\RightHeader{\節タイトル}
  }{%
    \addcontentsline{toc}{section}{\protect\numberline{\節番号}\節タイトル}
    \def\RightHeader{\節番号 \quad \節タイトル}
  }%
  \protected@edef\@currentlabelname{\節タイトル}
  \addvspace{\baselineskip}
}
% subsection
\newtcolorbox{subsectbox}{
  phantom={\phantomsection}, 
  blank,
  size=minimal,
  fontupper=\sf\large,
}
\RenewDocumentCommand{\subsection}{sm!t+}{
  \IfBooleanF{#1}{\refstepcounter{subsection}}
  \addvspace{\baselineskip}
  \begin{subsectbox}
    \IfBooleanTF{#1}
      {#2\hspace{.5em}\dotfill}
      {\小節番号 \quad #2\hspace{.5em}\dotfill}
  \end{subsectbox}
  \IfBooleanTF{#1}{%
    \IfBooleanT{#3}{\addcontentsline{toc}{subsection}{#2}}
  }{%
    \addcontentsline{toc}{subsection}{\protect\numberline{\小節番号}#2}
  }%
  \protected@edef\@currentlabelname{#2}
  \addvspace{.5\baselineskip}
}

% - - - 項目 - - -
\NewTColorBox[auto counter, number within=chapter]{none}{}{}
% \defitem{#1:環境名}{#2:日本語タイトル}{#3:色}[#4:下側コンテンツのタイトル]
\NewDocumentCommand{\defitem}{mmmO{}}{
  \crefname{thmlist#1}{#2}{#2}
  % \begin{#1}[##1:label][##2:nameref]"##3:サブタイトル"
  %   ...
  % \end{#1}
  \NewTColorBox[use counter from=none, crefname={#2}{#2}]{#1}{O{}O{}D""{}}{
    enhanced, breakable,
    boxrule=1pt, colframe=#3,
    colback=#3!10!white, skin=bicolor, colbacklower=white,
    fonttitle=\sf,
    nameref={##2},
    title={#2\thetcbcounter\IfBlankF{##3}{:##3}},
    IfBlankF={##1}{label={##1}},
    before upper={\crefalias{thmlisti}{thmlist#1}},
    before lower={\IfBlankF{#4}{\textsf{#4}}\quad},
  }
}
\newcommand{\tcbex}{\textsf{例}\quad}
% - - - 項目を定義 - - -
\defitem{elc}{解明}{myblue}[例]
\defitem{rgl}{規約}{myblue}[例]
\defitem{dfn}{定義}{myblue}[例]
\defitem{mdfn}{メタ定義}{myblue}[例]
\defitem{fct}{事実}{mygreen}[論証]
\defitem{lem}{補題}{mygreen}[証明]
\defitem{prp}{命題}{mygreen}[証明]
\defitem{cor}{系}{mygreen}[証明]
\defitem{mthm}{メタ定理}{mygreen}[証明]
\defitem{thm}{定理}{mygreen}[証明]
\defitem{req}{要請}{myred}[例]
\defitem{cnv}{約束}{mygray}[例]
\defitem{ntt}{記法}{mygray}[例]

% 番号のないもの
% \defitem{#1:環境名}{#2:日本語タイトル}{#3:色}[#4:下側コンテンツのタイトル]
\RenewDocumentCommand{\defitem}{mmmO{}}{
  % \begin{#1}"##1:サブタイトル"
  %   ...
  % \end{#1}
  \NewTColorBox{#1}{D""{}}{
    enhanced, breakable,
    boxrule=1pt, colframe=#3,
    colback=#3!10!white, skin=bicolor, colbacklower=white,
    fonttitle=\sf,
    title={#2\thetcbcounter\IfBlankF{##1}{:##1}},
    before upper={\crefalias{thmlisti}{thmlist#1}},
    before lower={\IfBlankF{#4}{\textsf{#4}\quad}},
  }
}
\defitem{axm}{公理}{myred}[例]
\defitem{infrule}{推論規則}{myred}[例]

% 例題環境を定義
\crefname{thmlistexc}{例題}{例題}
\NewTColorBox[auto counter, crefname={例題}{例題}]{exc}{O{}}{
  blank,
  enhanced, breakable,
  size=normal,
  borderline horizontal={1pt}{0pt}{myblue}, 
  title={例題\thetcbcounter},
  fonttitle=\sf, colbacktitle=myblue,
  attach boxed title to top left={xshift=-1pt, yshift=-\tcboxedtitleheight/2, yshifttext=-\tcboxedtitleheight/6},
  boxed title style={tile, size=small},
  IfBlankF={#1}{label={#1}},
  before upper={\crefalias{thmlisti}{thmlistexc}},
}

% 注意環境を定義
\NewTColorBox[auto counter, crefname={注意}{注意}]{nb}{O{}D""{}}{
  empty,
  enhanced, breakable,
  tile, colback=myyellow!10!white,
  borderline west={4pt}{0pt}{myyellow},
  detach title,
  fonttitle=\sf, coltitle=black,
  title={注意\thetcbcounter\IfBlankF{#2}{(#2)}},
  IfBlankF={#1}{label={#1}},
  before upper={\tcbtitle\quad},
}

% 汎用環境を定義
\NewTColorBox{misc}{O{}mD""{}}{
  blank,
  enhanced, breakable,
  size=normal,
  top=0pt, bottom=0pt,
  borderline west={2pt}{0pt}{mygray},
  detach title,
  fonttitle=\sf, coltitle=black,
  title={#2\IfBlankF{#3}{(#3)}},
  IfBlankF={#1}{label={#1}},
  before upper={\tcbtitle\quad},
}

% 概要環境を定義
\NewTColorBox{abst}{}{
  blank,
  enhanced, breakable,
  size=normal,
  borderline={.7pt}{0pt}{black,dashed},
  colback=white,
  title={\章番号 の概要},
  boxed title style={tile, size=small},
  fonttitle=\sf, coltitle=black, colbacktitle=white,
  attach boxed title to top text left={yshift*=-\tcboxedtitleheight/2},
  environment upper=itemize,
}



%%%%% \section{TikZ/グラフィックス}
\usepackage{forest}



%%%%% \section{表など}
\usepackage{xltabular}
\usepackage{booktabs}



%%%%% \section{索引}
% - - - \idx のkey-val - - -
\ExplSyntaxOn
\IfBoolTF{索引を出力}{
  \keys_define:nn { idx } {
    child .tl_set:N = \l_idx_child_tl,
    child-sort .tl_set:N = \l_idx_childsort_tl,
    entry .tl_set:N = \l_idx_entry_tl,
    entry-sort .tl_set:N = \l_idx_entrysort_tl
  }
  % - - - 索引コマンド - - -
  % \idx{#1:本文中の表示}{#2:よみがな(索引における#1のソート位置)}"#3:英訳"[#4:key-value リスト]
  % keyは以下の通り
  % child: 索引の子エントリのテキスト(初期値 = なし)
  % child-sort: child のソート位置(初期値 = child の値)
  % entry: 索引に実際に表示されるテキスト(初期値 = #1)
  % entry-sort: entry のソート位置(初期値 = entry の値)
  \NewDocumentCommand { \idx } { mmD""{}O{} } {
    \group_begin:
    \keys_set:nn { idx } { #4 }
    \emph{
      #1
      \tl_if_empty:nTF { #3 } {} { (#3) }
    }
    \tl_if_empty:NTF \l_idx_child_tl {
      \tl_if_empty:NTF \l_idx_entry_tl {
        \index{#2@#1}
      } {
        \index{\l_idx_entrysort_tl@l_idx_entry_tl}
      }
    } {
      \tl_if_empty:NTF \l_idx_entry_tl {
        \index{#2@#1!\l_idx_childsort_tl@\l_idx_child_tl}
      } {
        \index{\l_idx_entrysort_tl@l_idx_entry_tl!\l_idx_childsort_tl@\l_idx_child_tl}
      }
    }
    \group_end:
  }
  \usepackage{makeidx}
  % 索引の見出しを変更
  \patchcmd{\theindex}
    {\chapter*{\indexname}\addcontentsline{toc}{chapter}{\indexname}}
    {\chapter*{\indexname}+}{}{}
  \patchcmd{\begintheindex}{\begin{multicols}{2}}{\begin{multicols*}{2}}{}{}
  \patchcmd{\begintheindex}{\end{multicols}}{\end{multicols*}}{}{}
  % 記号類の見出しを変更
  \newcommand{\symbolindexname}{記号・数字}
  \makeindex
}{
  \NewDocumentCommand{\idx}{mmD""{}O{}}{\emph{#1\IfBlankF{#3}{(#3)}}}
  \renewcommand{\index}[1]{}
}
\ExplSyntaxOff



%%%%% \section{参考文献}
\IfBoolT{参考文献を出力}{
  \usepackage[style=authoryear, backend=biber]{biblatex}
  \addbibresource{biblio.bib}
  \DeclareNameAlias{default}{family-given}
  \newcommand{\PrintBibliography}{
    \nocite{*}
    \printbibliography[title=参考文献, label=biblio]
  }
}



%%%%% \section{グロッサリー}
\IfBoolTF{グロッサリーを出力}{
  \usepackage[record]{glossaries-extra}
  \GlsXtrLoadResources[src=symbols, set-widest, sort=use]
  \newglossarystyle{symbol-list}{
    \renewenvironment{theglossary}{
      \begin{longtable}[l]{@{}cll@{}}
    }{%
      \end{longtable}
    }
    \renewcommand*{\glossaryheader}{%
      \toprule
      {\sf 記号} & {\sf 説明} & {\sf 項目} \\
      \midrule
      \endfirsthead
      \toprule
      {\sf 記号} & {\sf 説明} & {\sf 項目} \\
      \midrule
      \endhead
      \bottomrule
      \endfoot
      \bottomrule
      \endlastfoot
    }
    \renewcommand*{\glossentry}[2]{%
      \glossentrysymbol{##1} & \glossentrydesc{##1} & \glstarget{##1}{\glossentryname{##1}} \tabularnewline
    }
    \renewcommand*{\glossarytitle}{記号一覧}
    \renewcommand*{\glossarytoctitle}{記号一覧}
    \newcommand{\PrintGlossary}{\printunsrtglossary[style=symbol-list]}
  }
}{
  \newcommand{\glsadd}[1]{}
}



%%%%% \section{その他}
% 章末問題
% 改行
\newcommand{\mypar}{\vspace{.5\baselineskip}\par}
% \emph の1番目の色
\DeclareEmphSequence{\color{myemphcolor}\sf}
% 例示
\newcommand{\example}[1]{\textsuperscript{\normalsize ∗}#1\textsubscript{\normalsize ∗}}

\makeatother
