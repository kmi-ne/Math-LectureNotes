%%%%% 真理値と条件分岐
\newcommand{\mysetbool}[2]{\csname #1#2\endcsname}
\mysetnewbool{BoolIfChapNumbered}{false}
\mysetnewbool{BoolIfSecNumbered}{false}
\mysetnewbool{BoolIfSecExist}{false}
\newcommand{\myboolTF}[3]{\csname if#1\endcsname #2 \else #3 \fi}
\newcommand{\myboolT}[2]{\csname if#1\endcsname #2 \else \fi}
\newcommand{\myboolF}[2]{\csname if#1\endcsname \else #2 \fi}



%%%%% ページレイアウト
\setpagelayout+{vmargin=10truemm, hmargin=17truemm, includemp, nomarginpar}
\usepackage[bottom]{footmisc}



%%%%% フォント/書体/文字関連
% - - - 和文 - - -
\ltjsetparameter{jacharrange={-2,-3}}
\setsansjfont{HaranoAjiGothic}[
  UprightFont={*-Medium},
  BoldFont={*-Bold},
]
% - - - 欧文/数学記号 - - -
\PassOptionsToPackage{math-style=ISO, colon=literal}{unicode-math}
\usepackage[olddefault, newcmbb, varnothing]{fontsetup}
\setmathfont[range={scr, bfscr}, StylisticSet=1]{NewCMMath-Regular}
\DeclareMathAlphabet{\cl}{OMS}{cmsy}{m}{n} % 旧チャンセリー筆記体
\setsansfont{nimbussans}
% - - - 追加数学記号 - - -
\usepackage{manfnt}
\usepackage[old]{old-arrows}
% - - - 数学記号調整 - - -
\usepackage{leftidx}
\usepackage{longmath}
% - - - テキスト書体 - - -
\newcommand{\gs}{\sffamily}
\newcommand{\gsb}{\gs\bfseries}
% - - - Unicode - - -
\usepackage{newunicodechar}
% - - - ルビ - - -
\usepackage{luatexja-ruby}



%%%%% 色
\usepackage{xcolor}
% - - - 色を定義 - - -
\definecolor{mygreen}{HTML}{75b651}
\definecolor{myblue}{HTML}{6e9dc2}
\definecolor{myred}{HTML}{c26e7c}
\definecolor{myyellow}{HTML}{b8ba05}
\definecolor{mygray}{HTML}{878787}
\definecolor{mywhitegray}{HTML}{cccccc}
\definecolor{myemphcolor}{HTML}{d11100}
\definecolor{myurlcolor}{HTML}{ad0d00}
\definecolor{mylinkcolor}{HTML}{2a4f99}
\definecolor{mycitecolor}{HTML}{2a4f99}
% - - - テキスト彩色用マクロ - - -
\newcommand{\gx}[1]{\textcolor{mygreen}{#1}}
\newcommand{\bx}[1]{\textcolor{myblue}{#1}}
\newcommand{\rx}[1]{\textcolor{myred}{#1}}
\newcommand{\Gx}[1]{\textcolor{mygray}{#1}}



%%%%% ヘッダー/フッター
% クラスファイルによるページスタイル強制変更を除去
\renewcommand{\plainifnotempty}{}
% - - - 部・章・節番号の出力形式を設定 - - -
\renewcommand{\postchaptername}{講}
\renewcommand{\presectionname}{§}
\newcommand{\presubsectionname}{§}
\makeatletter
\newcommand{\labelchapter}{\@chapapp\thechapter\@chappos}
\newcommand{\labelsection}{\@secapp\thesection\@secpos}
\newcommand{\labelsubsection}{\presubsectionname\thesubsection}
\makeatother
% - - - 専用ページスタイルの定義 - - -
\pagestyle{myheadings}
\markboth{lefthead}{righthead}
\usepackage{titleps}
\newpagestyle{myps}[\small\gs]{%
  \headrule
  \sethead[\textit{\thepage}][\myboolT{BoolIfChapNumbered}{\labelchapter\quad}\mychaptertitle][]
  {}{\myboolTF{BoolIfSecExist}
      {\myboolT{BoolIfSecNumbered}{\labelsection\quad}\mysectiontitle}
      {\myboolT{BoolIfChapNumbered}{\labelchapter\quad}\mychaptertitle}
  }{\textit{\thepage}}%
}
\pagestyle{myps}



%%%%% 見出しとセクショニング
% titleps の都合で AfterEndPreamble 実行
% - - - 章 - - -
\AfterEndPreamble{
  \NewCommandCopy{\oldchapter}{\chapter}
  \RenewDocumentCommand{\chapter}{sms}{
    \IfBooleanTF{#1}{
      \oldchapter*{#2}
      \IfBooleanT{#3}{\addcontentsline{toc}{chapter}{#2}}
      \mysetbool{BoolIfChapNumbered}{false}
    }{
      \oldchapter{#2}
      \mysetbool{BoolIfChapNumbered}{true}
    }
    \mysetbool{BoolIfSecExist}{false}
    \def\mychaptertitle{#2}
    \thispagestyle{empty}
  }
  % - - - 節 - - -
  \NewCommandCopy{\oldsection}{\section}
  \RenewDocumentCommand{\section}{sms}{
    \IfBooleanTF{#1}{
      \oldsection*{#2}
      \IfBooleanT{#3}{\addcontentsline{toc}{section}{#2}}
      \mysetbool{BoolIfSecNumbered}{false}
    }{
      \oldsection{#2}
      \mysetbool{BoolIfSecNumbered}{true}
    }
    \mysetbool{BoolIfSecExist}{true}
    \def\mysectiontitle{#2}
  }
  % - - - 小節 - - -
  \NewCommandCopy{\oldsubsection}{\subsection}
  \RenewDocumentCommand{\subsection}{sms}{
    \IfBooleanTF{#1}{
      \oldsubsection*{#2}
      \IfBooleanT{#3}{\addcontentsline{toc}{subsection}{#2}}
    }{
      \oldsubsection{#2}
    }
  }
}
% - - - セパレーター - - -
\usepackage{froufrou}



%%%%% 目次
\patchcmd{\tableofcontents}{*{\contentsname}}{*{\contentsname}\addcontentsline{toc}{chapter}{\contentsname}}{}{}
\setcounter{tocdepth}{2}
\makeatletter
\patchcmd{\l@subsection}{3.5\jsZw}{4\jsZw}{}{}
\makeatother



%%%%% ウォーターマーク/ラベル表示
\myboolT{BoolUseDraft}{
  \usepackage{draftwatermark}
  \usepackage[notref]{showkeys}
  \renewcommand*{\showkeyslabelformat}[1]{\normalfont\scriptsize\ttfamily\llap{\fbox{#1}}}
}



%%%%% ハイパーリンク
\usepackage{hyperref}
\hypersetup{
  setpagesize=false,
  bookmarksdepth=3,
  bookmarksnumbered=true,
  colorlinks=true,
  linkcolor=mylinkcolor,
  citecolor=mycitecolor,
  urlcolor=myurlcolor,
  pdfauthor={鴎海},
  linktoc=all
}
% - - - namerefで参照可能なラベル - - -
% \mylabel{#1:label}[#2:name]
\makeatletter
\NewDocumentCommand{\mylabel}{mO{}}{
  \bgroup
    \protected@edef\@currentlabelname{#2}
    \label{#1}
  \egroup
}
\makeatother
% - - - 任意の位置のラベル - - -
\newcommand{\phantomlabel}[1]{\phantomsection\label{#1}}
% - - - ページ番号が右上に表示されたハイパーリンク - - -
% \phyperref[#1:label]{#2:テキスト}
\newcommand{\phyperref}[2]{\hyperref[#1]{#2\textsuperscript{→p.\,\pageref*{#1}}}}



%%%%% cleveref
\usepackage[nameinlink]{cleveref}
% - - - ページ - - -
\crefname{page}{p.}{pp.}
% - - - 章 - - -
\makeatletter
\crefformat{chapter}{#2\@chapapp #1\@chappos #3}
\makeatother
% - - - 節 - - -
\crefname{section}{\presectionname}{\presectionname}
% - - - 小節 - - -
\crefname{subsection}{\presubsectionname}{\presubsectionname}
% - - - 数式 - - -
\crefname{equation}{式}{式}
\creflabelformat{equation}{#2(#1)#3}



%%%%% リスト
\usepackage{tasks}
\usepackage[inline]{enumitem}
\setlist{align=left, leftmargin=*}
% - - - thmlist 環境: theorem list - - -
% tcolorbox 内の項目のための enumerate 類似環境
\newlist{thmlist}{enumerate}{2}
\setlist[thmlist, 1]{label=\arabic{thmlisti}., ref=\thetcbcounter.\arabic{thmlisti}}
\setlist[thmlist, 2]{label=\arabic{thmlisti}.\arabic{thmlistii}., ref=\thetcbcounter.\arabic{thmlisti}.\arabic{thmlistii}}
% - - - myenum 環境: my enumerate - - -
\newlist{myenum}{enumerate}{3}
\setlist[myenum, 1]{label=(\arabic{myenumi}), ref=(\arabic{myenumi})}
\setlist[myenum, 2]{label=(\arabic{myenumi}.\arabic{myenumii}), ref=(\arabic{myenumi}.\arabic{myenumii})}
\setlist[myenum, 3]{label=(\arabic{myenumi}.\arabic{myenumii}.\arabic{myenumiii}), ref=(\arabic{myenumi}.\arabic{myenumii}.\arabic{myenumiii})}
% - - - description の item を ref と nameref で参照可能にしたもの- - -
% \begin{description}
%   \iteml[#1:項目名]{#2:label}[#3:nameref] 
% \end{description}
\makeatletter
\NewDocumentCommand{\iteml}{O{}mO{}}{
  \item[#1]
  \phantomsection
  \IfBlankTF{#3}
    {\renewcommand{\@currentlabel}{#1}\mylabel{#2}[#1]}
    {\renewcommand{\@currentlabel}{#3}\mylabel{#2}[#3]}
}
\makeatother



%%%%% 数式
\everymath=\expandafter{\the\everymath \narrowbaselines\displaystyle}
\usepackage{scalerel}
\allowdisplaybreaks[4]



%%%%% tcolorbox
\usepackage{tcolorbox}
\tcbuselibrary{skins, breakable}
\NewTColorBox[auto counter, number within=chapter]{none}{}{}
% \defitem{#1:環境名}{#2:日本語タイトル}{#3:色}[#4:下側コンテンツのタイトル]
\NewDocumentCommand{\defitem}{mmmO{}}{
  \crefname{thmlist#1}{#2}{#2}
  % \begin{#1}[##1:label][##2:nameref]"##3:サブタイトル"
  %   ...
  % \end{#1}
  \NewTColorBox[use counter from=none, crefname={#2}{#2}]{#1}{O{}O{}D""{}}{
    enhanced, breakable,
    boxrule=1pt, colframe=#3,
    colback=#3!10!white, skin=bicolor, colbacklower=white,
    fonttitle=\gsb,
    nameref={##2},
    title={#2\thetcbcounter\IfBlankF{##3}{:##3}},
    IfBlankF={##1}{label={##1}},
    before upper={\crefalias{thmlisti}{thmlist#1}},
    before lower={\IfBlankF{#4}{\textsf{#4}\hspace{1\zw}}},
  }
}
\newcommand{\tcbex}{{\gsb 例:}\hspace{1em}}
% - - - 項目を定義 - - -
\defitem{elc}{解明}{myblue}[例]
\defitem{rgl}{規約}{myblue}[例]
\defitem{dfn}{定義}{myblue}[例]
\defitem{mdfn}{メタ定義}{myblue}[例]
\defitem{fct}{事実}{mygreen}[論証]
\defitem{lem}{補題}{mygreen}[証明]
\defitem{prp}{命題}{mygreen}[証明]
\defitem{cor}{系}{mygreen}[証明]
\defitem{mthm}{メタ定理}{mygreen}[証明]
\defitem{thm}{定理}{mygreen}[証明]
\defitem{req}{要請}{myred}[例]
\defitem{cnv}{約束}{mygray}[例]

% 番号のないもの
% \defitem{#1:環境名}{#2:日本語タイトル}{#3:色}[#4:下側コンテンツのタイトル]
\RenewDocumentCommand{\defitem}{mmmO{}}{
  % \begin{#1}"##1:サブタイトル"
  %   ...
  % \end{#1}
  \NewTColorBox{#1}{D""{}}{
    enhanced, breakable,
    boxrule=1pt, colframe=#3,
    colback=#3!10!white, skin=bicolor, colbacklower=white,
    fonttitle=\gsb,
    title={#2\thetcbcounter\IfBlankF{##1}{:##1}},
    before upper={\crefalias{thmlisti}{thmlist#1}},
    before lower={\IfBlankF{#4}{\textsf{#4}\hspace{1\zw}}},
  }
}
\defitem{axm}{公理}{myred}[例]
\defitem{infrule}{推論規則}{myred}[例]

% 例題環境を定義
\NewTColorBox[auto counter, crefname={例題}{例題}]{exc}{O{}}{
  blank,
  enhanced, breakable,
  size=normal,
  borderline horizontal={1pt}{0pt}{myblue}, 
  title={例題\thetcbcounter},
  fonttitle=\gsb, colbacktitle=myblue,
  attach boxed title to top left={xshift=-1pt, yshift=-\tcboxedtitleheight/2, yshifttext=-\tcboxedtitleheight/6},
  boxed title style={tile, size=small},
  IfBlankF={#1}{label={#1}},
  before upper={\crefalias{thmlisti}{thmlist#1}},
}

% 汎用環境を定義
\NewTColorBox{misc}{O{}mD""{}}{
  blank,
  enhanced, breakable,
  size=normal,
  top=1pt, bottom=1pt,
  borderline west={2pt}{0pt}{mygray},
  detach title,
  fonttitle=\gsb, coltitle=black,
  title={#2\IfBlankF{#3}{(#3)}},
  IfBlankF={#1}{label={#1}},
  before upper={\tcbtitle\quad},
}

% 概要環境を定義
\NewTColorBox{abst}{}{
  blank,
  enhanced, breakable,
  size=normal,
  borderline={.7pt}{0pt}{dashed}, colframe=black,
  colback=white,
  title={\labelchapter の概要},
  boxed title style={tile, size=small},
  fonttitle=\gsb, coltitle=black, colbacktitle=white,
  attach boxed title to top text left={yshift*=-\tcboxedtitleheight/2},
  environment upper=itemize,
}



%%%%% TikZ・グラフィックス
\usepackage{forest}



%%%%% 表など
\usepackage{xltabular}
\usepackage{booktabs}



%%%%% 索引
% 記号類の見出しを変える
\newcommand{\symbolindexname}{記号・数字}
% - - - \idx のkey-val - - -
\ExplSyntaxOn
\myboolTF{BoolShowIndex}{
  \keys_define:nn { idx } {
    child .tl_set:N = \l_idx_child_tl,
    child-sort .tl_set:N = \l_idx_childsort_tl,
    entry .tl_set:N = \l_idx_entry_tl,
    entry-sort .tl_set:N = \l_idx_entrysort_tl
  }
  % - - - 索引コマンド - - -
  % \idx{#1:本文中の表示}{#2:よみがな(索引における#1のソート位置)}[#3:英訳][#4:key-value リスト]
  % keyは以下の通り
  % child: 索引の子エントリのテキスト(初期値 = なし)
  % child-sort: child のソート位置(初期値 = child の値)
  % entry: 索引に実際に表示されるテキスト(初期値 = #1)
  % entry-sort: entry のソート位置(初期値 = entry の値)
  \NewDocumentCommand { \idx } { mmD""{}O{} } {
    \group_begin:
    \keys_set:nn { idx } { #4 }
    \color{myemphcolor} \textsf{#1}
    \tl_if_empty:nTF { #3 } {} { (#3) }
    \tl_if_empty:NTF \l_idx_child_tl {
      \tl_if_empty:NTF \l_idx_entry_tl {
        \index{#2@#1}
      } {
        \index{\l_idx_entrysort_tl@l_idx_entry_tl}
      }
    } {
      \tl_if_empty:NTF \l_idx_entry_tl {
        \index{#2@#1!\l_idx_childsort_tl@\l_idx_child_tl}
      } {
        \index{\l_idx_entrysort_tl@l_idx_entry_tl!\l_idx_childsort_tl@\l_idx_child_tl}
      }
    }
    \group_end:
  }
  \usepackage{makeidx}
  \makeindex
}{
  \newcommand{\printindex}{}
  \NewDocumentCommand{\idx}{mmD""{}O{}}{\emph{#1}\IfBlankF{#3}{(#3)}}
}
\ExplSyntaxOff



%%%%% その他
% 章末問題
% 改行
\newcommand{\mypar}{\vspace{.5\baselineskip}\par}
% \emph の1番目の色
\DeclareEmphSequence{\color{myemphcolor}\sffamily}
% 例示
\newcommand{\example}[1]{\textsuperscript{\normalsize ∗}#1\textsubscript{\normalsize ∗}}
