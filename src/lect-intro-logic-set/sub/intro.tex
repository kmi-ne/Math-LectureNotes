\chapter*{はじめに(必ず読んでください)}+
\label{chap_intro}

\section*{本稿について}

本稿は,論理学と集合論への入門を目的とした,主に学部1年生向けの講義ノートです.

\section*{予備知識}

数学に関する予備知識は一切仮定しません.

\section*{本稿の書かれ方}

\begin{myenum}[itemsep=.5\baselineskip]
  \item 
  数学書は,プログラムのソースコードに似ています.プログラムのソースコードは,コードそのものとコメントとで構成されています.同様に,数学書も,\ltjkenten{数学的内容そのもの}と,\ltjkenten{それに対する説明}とで構成されていると思うことができます.
  \mypar

  ここで,数学的内容とは,定義・定理・証明など,数学そのものを構成する内容を指します.そして数学的内容の説明とは,例えば「なぜこの定義をする必要があるのか」「この定理は何を意味しているのか」など,その数学的内容の理解を助ける,あるいは補足情報を与えるために書かれる文章を指します.
  \mypar

  以下で示すように,本稿では,この2つをそれぞれ\emph{項目}と\emph{説明文}という形で区分し,一貫してそれらの徹底した分離を行います.そのため,本稿の内容は,項目のみをたどることで完全に完結するようになっています.純粋に論理的に言えば,\ltjkenten{説明文を読む必要は一切ありません}.しかし,私たちは機械ではないので,長大なソースコードをコメントなしで理解することが困難であるように,説明文を本当に一切読まずに理解することは困難でしょう.
  \mypar

  本稿の全ての数学的内容は,次のような色枠付きのボックスに収められています.
  \begin{thm}
    ここに定理が入ります.
    \tcblower
    ここに証明が入ります.
  \end{thm}
  このようなボックスを,本稿では\emph{項目}と呼びます.

  全ての項目には,各々の役割に応じた色が以下のように割り当てられています.
  \begin{longtable}{lll}
    \hline
    {\sf 色} & {\sf 役割} & {\sf 項目名} \\
    \hline\hline
    \endfirsthead
    \endhead
    \endfoot
    \hline
    \endlastfoot
    \青{$◾$ \textsf{青色}} & 定義タイプ & 規約,メタ定義,定義 \\
    \赤{$◾$ \textsf{赤色}} & 公理タイプ & 公理,推論規則 \\
    \緑{$◾$ \textsf{緑色}} & 定理タイプ & 事実,メタ定理,定理,補題,命題 \\
    \灰{$◾$ \textsf{灰色}} & 非形式的な約束 & 記法,約束
  \end{longtable}

  一方,数学的内容に対する説明は,色付きボックスには収められず,この文と同じような普通の文章として記載されます.このような文章を,本稿では\emph{説明文}と呼びます.説明文はその内容に応じて,様々な箇所に書かれます.
  \mypar

  特に,「\emph{導入}」と記されている§は,完全に説明文のみで構成されている§です.これは,その後ろに続く「\emph{本論}」という§の全体を解説する目的で書かれています.

  \item 
  本稿は,前から順に通読できるよう書かれています.特に,本稿の任意の項目は,それ以前に既に記載している項目のみを既知として書かれています.

  \item 
  初読では理解が困難であったり,一読しただけでは誤解が予想されるために特に注意深く読むべき部分には,$\textdbend$を表示します.$\textdbend$は,該当部分の開始位置の左側余白と,終了位置の右側余白に表示されます.
\end{myenum}
