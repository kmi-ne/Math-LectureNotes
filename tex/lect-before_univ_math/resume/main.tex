\documentclass[lualatex, ja=standard, b5j, base=9pt, label-section=none]{bxjsarticle}

\ltjsetparameter{jacharrange={-2,-3}}
\PassOptionsToPackage{sans-style=literal, bold-style=literal}{unicode-math}
\usepackage{fontsetup}
\setsansfont{nimbussans}
\def\sf{\sffamily}
\DeclareEmphSequence{\sf}

\usepackage{luatexja-ruby}
\def\kt{\ltjkenten}

\usepackage{multicol}

\title{\sf 速習コース「大学で数学を学ぶ前に」実施要綱}
\author{鴎海}
\date{\normalsize 最終更新日:\today}

\begin{document}

\maketitle


\begin{multicols*}{2}

  \section{概要}

  速習コース「大学で数学を学ぶ前に」は,主に理工系の大学1年生に向けた講義です.また,数学的な基礎力を固めたい大学2年生以上にも適しています.この講義は,大学で微分積分や線形代数を学ぶ前に習得しておくべき基礎的な数学的論証法や数学的知識を,演習を交えながら,短期間で体系的に身に付けることを目的としています.


  \section{ねらい}

  \begin{itemize}
    \item 数学的主張・証明の論理構造を読み取る力をつける.
    \item 論理と集合の言葉を習得する.
    \item 以上によって,正確な数学的議論を,自信をもって,しかも見通しよく行う力をつける.
    \item 以上を基盤として,B1の数学の基礎部分へとシームレスに接続する.
  \end{itemize}


  \section{内容}

  主な内容は以下の通りです.
  \begin{itemize}
    \item 論理と証明
    \item 集合・写像
    \item 実数論
    \item B1の解析学や線形代数学の基礎部分
  \end{itemize}

  数学に関する予備知識は特に仮定しません.


  \section{実施要綱}

  \begin{description}
    \item[形式] 講義
    \item[日時] 毎週木曜18:30--(初回:5/8)
    \item[場所] 理学部H棟7階\ コミュニケーションスペース
  \end{description}


  \section{参加方法}

  特にこれといった参加申請などはありませんが,できれば全体LINE等で私が送った開催告知にリアクションを付けてもらえると助かります(おおよその人数を把握したいので).


  \section{開講にあたって}

  数学それ自体を研究するにせよ(数学科),数学のユーザーとなるにせよ(数学科以外),数学は理工系の\emph{共通言語}です.人が言語を習得するように,数学を習得することはまさに一種のリテラシーなのです.だからこそ,その共通言語で正しく読み書きを行う力,つまり,\emph{数学的議論を正しく理解し実行する力}が必要となります.

  そうしたリテラシーを身に付けるうえで,できるだけ早い段階で習得しておきたいのが,\emph{論理}と\emph{集合}の言葉です.というのも,大学数学,そして現代のほぼ\emph{全ての数学}は,この論理と集合を基盤として統一的に構築されており,これらを十分知っていることが前提となっているからです.逆に,大学以降の数学を論理や集合の知識なしで十分に理解するのは,たとえ微分積分や線形代数であっても,かなりの困難を伴います.それはさながら,人称や時制といった基本的な文法を知らないのに,複雑な英文を読み書きしようとするようなものです.

  しかし残念ながら,こうした知識のほとんどは,大学の講義では十分体系的に扱われない傾向にあります.つまり,説明しなくてもそのくらい\kt{感覚的}・\kt{常識的に}分かるだろう,という暗黙の了解で事が進んでしまうか,良くて講義の合間合間にアドホックな説明が入るにとどまるか,であることが多いのです.

  そこで,この講義では,大学数学を学ぶ前に習得しておくべきこれらのトピック --- 論理と集合による数学の展開 --- を,一から体系的に解説します.それは,数学を行う上でのスタートラインに立つというだけでなく,数学の「しくみ」を俯瞰し,結果的に思考の整理・節約へとつなげることでもあります.

  また,こうした知識を内面化してスムーズに数学的議論を行えるようになるためには,実際に手を動かして練習することが重要です.そのため,本講義では,問題演習をできる限り豊富に交えます.

\end{multicols*}

\end{document}
